\documentclass{article}
\usepackage[margin=1in]{geometry} % smaller margins

\title{CS273A Project Report}
\author{The Vectors Anonymous Support Group\\Kyle Benson, Eugenia Gabrielova}
\date{December 2013}
\begin{document}
\maketitle

%TODO fancy text formatting for scikit learn to emphasize it with the funny capitalization?


\section{Introduction}

%TODO

%%%%%%%%%%%%%%%%%%%%%%%%%%%%%%%%%%%%%%%%%%%%%%%%%%%%%%%%%%%%%%%%%%%%%%

\section{Platform}

This section explains which software libraries and tools we made use of and why we chose them.
See Section \ref{implementation} for how we used them.

\subsection{General}
%TODO: this sounds like implementation details... move?
We chose to implement this project in Python as we are both more familiar with it and prefer it over MATLAB\textregistered.
For more efficient vector and matrix representations than standard lists, we naturally used the numpy library, as does the two machine learning libraries that we used.

To facilitate team-oriented development, we set up a repository on GitHub to easily merge incremental changes and improvements to our code as well as track task completion.

\subsection{scikit learn}

%TODO anything to add, @Eugenia?

To learn the data and make predictions we used scikit learn \cite{pedregosa2011scikit}, one of the more mature Python-based machine learning libraries.
It provides a fairly comprehensive list of classes implementing a common Classifier interface.
This library also provides utility functions for working with data such as shuffling, splitting for cross-validation, and imputing data as well as metrics for scoring classifiers.

\subsection{PyBrain}

Neural networks currently are not fully supported in scikit learn; so we used PyBrain instead \cite{schaul2010}.
PyBrain allows for programmatically creating arbitrary neural network structures, including multiple layers of hidden nodes.
These networks are trained on numpy arrays of data using various training algorithms provided in the package.

%%%%%%%%%%%%%%%%%%%%%%%%%%%%%%%%%%%%%%%%%%%%%%%%%%%%%%%%%%%%%%%%%%%%%%

\section{Implementation}
\label{implementation}

This section describes our design choices and reasoning for the software we wrote.

\subsection{Design Overview}

We implemented each classifier as a separate Python script that:
\begin{enumerate}
\item Loads the KDD physics data set
\item Massages it into an appropriate format
\item Trains a Classifier object on the data
\item Prints scores for training and test data

%TODO:cross-validation?

\end{enumerate}

We also created a few utility files for interfacing with some of those from scikit learn.
These also include tasks such as outputting predictions on the test data in the proper format.

%TODO: @Eugenia, anything to add?

\subsection{Tuning Hyper-Parameters}

%TODO:Eugenia's approach?

%TODO:@Kyle, GridSearchCV

\subsection{Using Multiple Libraries}

We mostly used scikit learn for learning the data; so our general design approach closely followed that of scikit learn, including the use of numpy arrays and matrices.
To fit all our Classifiers to a common API, we wrote an adaptor class for a NeuralNetworkClassifier so that we could use PyBrain structures with the scikit learn Classifier interface.
This made using code applied to Classifiers from both packages easier to share between files and classifiers.
For example, we used the GridSearchCV feature from scikit learn to quickly explore different parameters for the neural network training algorithm.
This approach worked especially well with proper version controlling as each team member could separately develop new features and then merge them without modifications due to working with different libraries.

%TODO:@Eugenia, too verbose in technical detail here?

%%%%%%%%%%%%%%%%%%%%%%%%%%%%%%%%%%%%%%%%%%%%%%%%%%%%%%%%%%%%%%%%%%%%%%

\section{Data Preprocessing}
%TODO
Imputing missing data
removing missing data

%%%%%%%%%%%%%%%%%%%%%%%%%%%%%%%%%%%%%%%%%%%%%%%%%%%%%%%%%%%%%%%%%%%%%%

\section{Classifiers}
%TODO
\subsection{Random Forests}

\subsection{AdaBoost}

\subsection{Gradient Boost}

\subsection{Logistic Regression}

\subsection{Neural Networks}


%%%%%%%%%%%%%%%%%%%%%%%%%%%%%%%%%%%%%%%%%%%%%%%%%%%%%%%%%%%%%%%%%%%%%%

\section{Feature Selection}
%TODO
how we did it
why we did it
  missing data

%%%%%%%%%%%%%%%%%%%%%%%%%%%%%%%%%%%%%%%%%%%%%%%%%%%%%%%%%%%%%%%%%%%%%%

\section{Other stuff}
%TODO
2004 winners' report \cite{vogel2004anti}

on ensembling \cite{caruana2004ensemble}

another kdd '04 perspective \cite{caruana2004kdd}

weka approach, which we didn't choose \cite{pfahringer2004weka}

%%%%%%%%%%%%%%%%%%%%%%%%%%%%%%%%%%%%%%%%%%%%%%%%%%%%%%%%%%%%%%%%%%%%%%

\section{Results}
%TODO
\subsection{Accuracy}

%%%%%%%%%%%%%%%%%%%%%%%%%%%%%%%%%%%%%%%%%%%%%%%%%%%%%%%%%%%%%%%%%%%%%%

%TODO: do we need a conclusion?

\bibliographystyle{IEEEtran}
\bibliography{report}

\end{document}
