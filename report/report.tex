\documentclass{article}
\usepackage[margin=1in]{geometry} % smaller margins

\title{CS273A Project Report}
\author{The Vectors Anonymous Support Group\\Kyle Benson, Eugenia Gabrielova}
\date{December 2013}
\begin{document}
\maketitle

\section{Introduction}

The Knowledge Discovery and Data Mining (KDD) Cup in 2004 included a binary classification problem for a particle generated in a high energy collider.
Data on 78 attributes was collected and collated into a training set of 50,000 points and a validation set of 100,000 points.
Our team's task was to identify a classifier, and its associated hyper-parameters, to maximize classification on the test data with regards to one of four metrics (accuracy, area under curve, log loss, and SLQ).

This report outlines our approach to this problem, what software we used, our design choices for the code we wrote to accomplish this task, which classifiers we used, and the results of our work. We conclude with a discussion of trade-offs in our process and considerations for future work on this data. 

%%%%%%%%%%%%%%%%%%%%%%%%%%%%%%%%%%%%%%%%%%%%%%%%%%%%%%%%%%%%%%%%%%%%%%

\section{Platform}

This section explains which software libraries and tools we made use of and why we chose them.
See Section \ref{implementation} for how we used them.

\subsection{Project Organization}
We chose to implement this project in Python as we are both more familiar with it and prefer it over MATLAB\textregistered.
For more efficient vector and matrix representations than standard lists, we naturally used the numpy library, as do the two machine learning libraries that we used.

To facilitate team-oriented development, we set up a repository on GitHub to easily merge incremental changes and improvements to our code as well as track task completion.

\subsection{scikit-learn}

%TODO anything to add, @Eugenia?

To learn the data and make predictions we used scikit-learn \cite{pedregosa2011scikit}, one of the more mature Python-based machine learning libraries.
It provides a fairly comprehensive list of classes implementing a common Classifier interface.
This library also provides utility functions for working with data such as shuffling, splitting for cross-validation, and imputing data as well as metrics for scoring classifiers.

\subsection{PyBrain}

Neural networks currently are not fully supported in scikit-learn; so we used PyBrain instead \cite{schaul2010}.
PyBrain allows for programmatically creating arbitrary neural network structures, including multiple layers of hidden nodes.
These networks are trained on numpy arrays of data using various training algorithms provided in the package.

%%%%%%%%%%%%%%%%%%%%%%%%%%%%%%%%%%%%%%%%%%%%%%%%%%%%%%%%%%%%%%%%%%%%%%

\section{Implementation}
\label{implementation}

This section describes our design choices and reasoning for the software we wrote.

\subsection{Design Overview}

We implemented each classifier as a separate Python script that:
\begin{enumerate}
\item Loads the KDD physics data set
\item Massages it into an appropriate format
\item Trains a Classifier object on the data
\item Prints scores for training and test data

%TODO:cross-validation?

\end{enumerate}

We also created a few utility files for interfacing with some of those from scikit-learn.
These also include tasks such as outputting predictions on the test data in the proper format.

%TODO: @Eugenia, anything to add?

\subsection{Tuning Hyper-Parameters}

%TODO:@Eugenia's approach?
%TODO:tuning for various metrics?

To tune the hyper-parameters of our classifiers, we explored many combinations at once and compared the results from each.
From these, we chose the best parameters and explored slight modifications to them further, iterating this process several times until we saw no change in performance.
The scikit-learn class GridSearchCV helped in this process.
We defined a grid of possible parameters and it ran the given classifier on every combination of them with 3-fold cross-validation.
It also provides an easy method for running multiple jobs in parallel, speeding up the exploration of each parameter space and reducing the time between tuning iterations.

\subsection{Using Multiple Libraries}

We mostly used scikit-learn for learning the data; so our general design approach closely followed that of scikit-learn, including the use of numpy arrays and matrices.
To fit all our Classifiers to a common API, we wrote an adaptor class for a NeuralNetworkClassifier so that we could use PyBrain structures with the scikit-learn Classifier interface.
This made using code applied to Classifiers from both packages easier to share between files and classifiers.
For example, we used the GridSearchCV feature from scikit-learn to quickly explore different parameters for the neural network training algorithm.
This approach worked especially well with proper version controlling as each team member could separately develop new features and then merge them without modifications due to working with different libraries.

%TODO:@Eugenia, too verbose in technical detail here?

%%%%%%%%%%%%%%%%%%%%%%%%%%%%%%%%%%%%%%%%%%%%%%%%%%%%%%%%%%%%%%%%%%%%%%

\section{Data Preprocessing}

	One challenge

Imputing missing data
removing missing data

%%%%%%%%%%%%%%%%%%%%%%%%%%%%%%%%%%%%%%%%%%%%%%%%%%%%%%%%%%%%%%%%%%%%%%

\section{Classifiers}
%TODO
\subsection{Random Forests}

\subsection{AdaBoost}

\subsection{Gradient Boost}

\subsection{Logistic Regression}

\subsection{Neural Networks}


%%%%%%%%%%%%%%%%%%%%%%%%%%%%%%%%%%%%%%%%%%%%%%%%%%%%%%%%%%%%%%%%%%%%%%

\section{Feature Selection}
%TODO
how we did it
why we did it
  missing data

%%%%%%%%%%%%%%%%%%%%%%%%%%%%%%%%%%%%%%%%%%%%%%%%%%%%%%%%%%%%%%%%%%%%%%

\section{Other stuff}
%TODO
2004 winners' report \cite{vogel2004anti}

on ensembling \cite{caruana2004ensemble}

another kdd '04 perspective \cite{caruana2004kdd}

weka approach, which we didn't choose \cite{pfahringer2004weka}

%%%%%%%%%%%%%%%%%%%%%%%%%%%%%%%%%%%%%%%%%%%%%%%%%%%%%%%%%%%%%%%%%%%%%%

\section{Results}
%TODO
\subsection{Accuracy}

%%%%%%%%%%%%%%%%%%%%%%%%%%%%%%%%%%%%%%%%%%%%%%%%%%%%%%%%%%%%%%%%%%%%%%

\section{Contributions}

Eugenia implemented preprocessing functionality (imputation and feature selection) and tested 10 classifiers (without added complexity) with cross-validation on the testing data. 
After finding the 5 most powerful base classifiers (those with > .7 training accuracy), she added some complexity parameters combined them into an unweighted majority-vote ensemble. 
After observing remarkable accuracy in the gradient boosting classifier for this data set, Eugenia decided to focus on its optimization.
Though she tried to optimize both for area under curve and accuracy, she found that the classifier performed better with accuracy and focused complexity updates on that metric.

Kyle was curious about the potential success of a Neural Network for classification of this data. 
He compared neural network options in the scikit-learn library with external libraries and selected pybrain to complete his work.
He implemented a deep learner and added a wrapper to that it could be called with the same interface as the sci-kit learn classifiers.
Kyle used the "Top 5" classifiers as features of a logistic classifier and evaluated a more weighted performance of the stacked ensembles.

%%%%%%%%%%%%%%%%%%%%%%%%%%%%%%%%%%%%%%%%%%%%%%%%%%%%%%%%%%%%%%%%%%%%%%

\section{Conclusion}
%TODO

%%%%%%%%%%%%%%%%%%%%%%%%%%%%%%%%%%%%%%%%%%%%%%%%%%%%%%%%%%%%%%%%%%%%%%

\bibliographystyle{IEEEtran}
\bibliography{report}

\end{document}
